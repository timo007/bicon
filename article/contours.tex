\documentclass[12pt,a4paper]{article}

\usepackage{natbib}
\usepackage{graphicx}
\usepackage{url}
\usepackage{siunitx}
\usepackage[margin=1.5cm]{geometry}
\usepackage{lineno}
\usepackage{parskip}
\usepackage{setspace}
\usepackage{fontspec}
\usepackage{polyglossia}

% The fakaʻua character (many fonts don't have it).
\newcommand{\f}{{\fontspec{LibertinusSerif}ʻ}}

\doublespacing

\title{Contour-based compression of numerical weather prediction data}
\author{Timothy Hume}

\begin{document}

\maketitle


\linenumbers
\begin{abstract}
\begin{center}
\includegraphics[width=6cm]{figures/abstract.pdf}
\end{center}
A contour-based compression method for numerical weather prediction data is
described. Very high compression ratios of over fifty are achievable for some
meteorological fields. Potential applications of the method include
dissemination of weather forecast data to users with low bandwidth or expensive
communication systems.
\end{abstract}

\section*{Introduction}
\label{sec:introduction}

Compression of meteorological data is becoming increasingly important. There
has been an explosion in the volume of meteorological data collected and
produced in recent years. For example, the European Centre for Medium-Range
Weather Forecasts (ECMWF) has found that their data archives are increasing at
such a rapid rate that storage is becoming expensive and
difficult~\citep{burton_cruz2020}. Compression is one method which can help
tackle this explosion of data.

Compression is also used in applications where dissemination of data is limited
by low bandwidth communication channels. With the development of faster and
cheaper data transmission networks, this problem is not as acute as in the
past. However, there are still users who are limited by slow or expensive
communication.

A contour-based compression method called Binary Contours (BinCon) is
presented in this article. It achieves very high compression ratios when
applied to numerical weather prediction (NWP) data. Rather than
compressing grid point data, BinCon contours the meteorological fields
and saves the contours as line segments in a compact binary format. Gridded
fields can subsequently be recreated from these contours. Contour-based
compression methods have previously been used for geospatial images, radar
images, and weather maps. However, the focus has often been on image data
rather than NWP data~\citep[see][]{scarmana2011,
mahapatra_makkapati2005, gertz_grappel1994}

One of the earliest contour-based compression methods predates the
computer era. International Analysis Code (IAC~FLEET) is a standard
World Meteorological Organization (WMO) alphanumeric code for encoding
weather maps. It was adopted for use by the International Meteorological
Organization in 1947~\citep[Resolution 156, p.~161]{imo1947}. It is used for
marine applications where communications have historically been slow and
expensive~\citep[pp.~87--90]{wmo2019a}. A typical IAC~FLEET message is
between about one and three thousand characters in length. Examples of current
IAC~FLEET analyses for the Southwest Pacific and North Atlantic,
produced by the Fiji Meteorological Service (FMS) and National
Oceanographic and Atmospheric Administration (NOAA) respectively, can
be downloaded from the Internet~\citep{fms2023, noaa2023}. A decoded and
plotted analysis is shown in Figure~\ref{fig:iacplot}.

\begin{figure*}
\centering
\includegraphics[width=11.7cm]{figures/iacplot.pdf}
\caption{\label{fig:iacplot}Isobars plotted from Fleet Code using
OpenCPN~\citep{opencpn2023} The yellow lines are easterly wind troughs, and the
orange line is a frontal system.}
\end{figure*}

Although IAC~FLEET is still an official code, it has low spatial resolution.
With the advent of high-resolution Numerical Weather Prediction (NWP) models
and faster communications methods, IAC~FLEET is rarely used.

BinCon is inspired by IAC~FLEET. It enables high-resolution fields to be
encoded into files of similar size to IAC~FLEET. Furthermore, any NWP field can
be encoded, not just mean sea level pressure isobars. For example,
Figure~\ref{fig:BinCon} shows mean sea level pressure compressed using BinCon
for the New Zealand region\footnote{The New Zealand region in this article
refers to the region from \ang{142}\,E--\ang{164}\,W,
\ang{54}\,S--\ang{27}\,S.}. Data are from the NCEP (National Centers for
Environmental Prediction) GFS (Global Forecast System) model. The compressed
data file used to make this plot is just \SI{1845}{\byte} in size.

\begin{figure*}
\centering
\includegraphics[width=11.7cm]{figures/example.pdf}
\caption{\label{fig:BinCon}Mean sea level pressure forecast valid at
1200\,UTC, 5 January 2023. Data are compressed using the BinCon
method.}
\end{figure*}

\section*{Method}
\label{sec:method}

The BinCon method has three steps: contour the raw gridded data, simplify the
contours, and convert the simplified contour locations to a binary format.
These are described in the following section.

\subsection*{Simplified contour maps}
\label{sec:contouring}

The first step of the compression process is to create a contour plot of the
gridded data. For example, Figure~\ref{fig:nzmslp}\,(a)
shows a contour plot of a mean sea level pressure forecast for the New Zealand
region using data from the NCEP GFS model. The spatial resolution of the
original gridded data is \ang{0.25}. The forecast has a lead time of
\SI{24}{\hour}, and is valid at 1200\,UTC, 5 January 2023. These forecast data are
used throughout this article.

The appropriate contour spacing depends on the field being compressed. For
smoothly varying fields such as mean sea level pressure, a rather coarse
spacing may be adequate. For fields with large gradients, smaller spacings may
be required to adequately resolve the field in areas where the gradient is
large. There is no requirement for the contour spacing to be uniform. For
example, wind speed isotachs might be closer together for low wind speeds than
for higher wind speeds.

At this stage, the contour maps can be quite detailed. For example, in
Figure~\ref{fig:nzmslp}\,(a) numerous small wiggles are observed in the
isobars. It is apparent that these could be simplified without losing much
information of practical importance.

To simplify the contours, the Ramer-Douglas-Peucker algorithm is
used~\citep{ramer1972, douglas_peucker1973}. This algorithm operates by
successively removing points from a complex polygonal line, provided the
distance of the resulting line is no further away from the original line than a
user specified tolerance. An interactive demonstration of the algorithm can be
found in~\citet{karthaus2012}.

Figures~\ref{fig:nzmslp}(b)--(d) show the effect of the tolerance setting on
the simplified contours. When the tolerance is \ang{0.125}, the simplified
contours are very close to the original contours. This is not surprising given
the original grid spacing is \ang{0.25}, and a simplified contour falling
within \ang{0.125} of the original contour is likely to be in the same grid
box. As the tolerance increases, the simplified contours become cruder. The
appropriate tolerance will depend on the use to which the data are intended.
Larger tolerances produce smaller compressed file sizes.

\begin{figure*}
\centering
\includegraphics[width=18cm]{figures/nzmslp.pdf}
\caption{\label{fig:nzmslp}Mean sea level pressure forecast valid at
1200\,UTC, 5 January 2023.}
\end{figure*}

\subsection*{Binary contour format}

We now have a simplified contour map of the field we are processing. There are
one or more contours, each comprised of one or more line segments. The line
segments are defined by the geographical locations of their starting and ending
points.

The data are encoded into a header record, and one contour record for each
discrete contour line on the map. The header record contains meta-data
describing the meteorological field being encoded (see
Table~\ref{tab:header_record}). The contour records contain the locations of
the line segments making up the contours (see Table~\ref{tab:contour_record})
\footnote{The data types in Tables~\ref{tab:header_record}~and~\ref{tab:contour_record}
are: UInt8 and UInt16 are 8 and 16-bit unsigned integers; SInt8 are 8-bit
signed integers; Float32 and Float64 are IEEE 32 and 64-bit floating-point
numbers; Char is a one byte character.}.

The procedure for encoding starts with the longitude and latitude
of the first point in a contour, expressed as 32-bit IEEE floating-point
numbers. The location of the next point is the starting point plus offsets in
the longitude and latitude dimensions in multiples of a spatial increment,
$\delta$. For example, if the first two points in a contour are at
(\ang{135.25}\,E,~\ang{-45.50}\,N) and (\ang{135.75}\,E,~\ang{-45.75}\,N), and
$\delta = \ang{0.01}$, the offset of the second point from the first point will
be (50, -25). Offsets are always relative to the preceding point on the contour
line. Fractional offsets are rounded to the nearest integer. The value of
$\delta$ is chosen so that no offsets in the contour line fall outside the
range $[-128, 127]$. This ensures the offsets can be represented as 8-bit signed
integers. This offset procedure continues until the last point in the contour
is reached. Then a new record is commenced for the next contour in the map
being encoded.

Readers may have noticed that the rounding of offsets to the nearest integer
can introduce a source of error in the encoding of the contour lines. In
practice these errors are small, and typically not visible when decoded
contours are plotted on a map.

All data are written to file in network-endian format. This ensures there is no
ambiguity when encoding and decoding data on either big or little-endian
computers. 

\begin{table*}
\centering
\begin{tabular}{llll} \hline
Byte range  & Type      & Description                    & Units \\ \hline
1           & UInt8     & WMO discipline code            & WMO code table~\citep[Table 0.0]{wmo2014}  \\
2           & UInt8     & WMO category code              & WMO code table~\citep[Table 4.1]{wmo2014}  \\
3           & UInt8     & WMO parameter code             & WMO code table~\citep[Table 4.2]{wmo2014}  \\
4--11       & Float64   & NWP base time                  & Seconds since 0000\,UTC, 1 January 1970    \\
12--15      & Float32   & Forecast lead time             & Hours since base time                      \\ 
16--23      & Char      & Vertical level                 & e.g. 950hPa, MSL, surface                  \\
24--27      & Float32   & Eastern boundary of domain     & degrees East                               \\
28--31      & Float32   & Western boundary of domain     & degrees East                               \\
32--35      & Float32   & Southern boundary of domain    & degrees North                              \\
36--39      & Float32   & Northern boundary of domain    & degrees North                              \\ \hline
\end{tabular}
\caption{\label{tab:header_record}Format of the header record.}
\end{table*}

\begin{table*}
\centering
\begin{tabular}{llll} \hline
Byte range     & Type      & Description                                      & Units or value                                         \\ \hline
1--4           & Float32   & Contour level (e.g. \SI{100000}{\Pa})            & Variable dependent                                     \\
5--6           & UInt16    & $n$: number of points in the contour line        &                                                        \\
7--10          & Float32   & $\delta$: spatial increment                      & degrees                                                \\
11--14         & Float32   & $\lambda_1$: longitude of first point in contour & degrees East                                           \\
15--18         & Float32   & $\phi_1$: latitude of first point in contour     & degrees North                                          \\
19             & SInt8     & Longitude offset from previous point             & $\lfloor(\lambda_2 - \lambda_1) / \delta\rceil$        \\
20             & SInt8     & Latitude offset from previous point              & $\lfloor(\phi_2 - \phi_1) / \delta\rceil$              \\
\vdots         & \vdots    & \vdots                                           & \vdots                                                 \\
$19 + 2(n-1)$  & SInt8     & Longitude offset from previous point             & $\lfloor(\lambda_n - \lambda_{(n-1)}) / \delta\rceil$  \\
$20 + 2(n-1)$  & SInt8     & Latitude offset from previous point              & $\lfloor(\phi_n - \phi_{(n-1)}) / \delta\rceil$        \\ \hline
\end{tabular}
\caption{\label{tab:contour_record}Format of a record encoding a single contour line.}
\end{table*}

\subsection*{Decoding contours}
\label{sec:decoding}

Decoding of encoded contours is a simple reversal of the procedure used for
encoding. Once decoded, the contour lines may be plotted on a map.

Some users will prefer gridded data to contoured data. For each line segment in
the contour data, the contour value and the location of the end points are
taken. This provides an irregular network of data points. Sometimes more than
one of these data points will fall within a grid cell in the output grid. When
this happens, the average value and location of the points which fall within
the grid cell is taken. Finally, a two-dimensional surface on the output grid
is fitted to the irregular network of points using the method of continuous
curvature splines in tension described by \cite{smith_wessel1990}. The coloured
grid cells in Figure~\ref{fig:nzmslp} were created from these gridded data.

\section*{Compression performance}
\label{sec:performance}

The compression method described in the previous section is a lossy scheme;
compressed and then decompressed data will differ from the original data. There
are two adjustable parameters which affect how much the data are compressed by,
and how much information is lost. These are the spacing when contouring the
original gridded data, and the tolerance when simplifying the resulting
contours.

To investigate the effect of the settings on the performance of the
compression, the mean sea level pressure forecast for the New Zealand region
used in the previous section was compressed with various contour spacings and
tolerances. The resulting data were then decompressed, gridded and compared
with the original field. The results are shown in
Figures~\ref{fig:nzmslp}(e)--(g) and~\ref{fig:nzmslp}(j)--(k).

As the isobar spacing increases, so too does the root mean square error (RMSE)
of the compressed field when compared to the original field. On the other hand,
the file sizes get a lot smaller with coarse contour spacings. Similarly, as
the tolerance increases, the RMSE increases and the file sizes decrease.

The appropriate settings for the contour interval and tolerance will depend on
the forecast user's requirements. The errors introduced by the compression will
also depend on the fields being compressed. Some fields will compress better
than others; this is discussed later.

\subsection*{Comparison with other compressed data formats}
\label{sec:compression_comparison}

Many data formats have compression incorporated into them. The WMO standard for
transmission of gridded binary data is GRIB2. GRIB2 usually employs a simple
lossless packing method, but it can also use a lossy JPEG~2000-based method,
amongst others~\citep[p. 143]{wmo2019b}. NetCDF is another commonly used
scientific data format. It utilises zlib-based lossless compression.

To set a baseline from which to compare the various data compression formats,
we consider uncompressed data to be represented by one 32-bit IEEE
floating-point number for each grid point value in the domain. Some fields may
be adequately represented by 16-bit half precision floats, and thus use half as
much disc space as 32-bit floats. Very few, if any, geophysical fields require
64-bit double precision floats.

The mean sea level pressure forecasts used previously were converted to the
various formats using \texttt{wgrib2}~\citep{noaa2021}.
Table~\ref{tab:compression_comparison} shows sizes of the files covering the
New Zealand region, along with the RMSE of the compressed data compared to the
original data.

\begin{table*}
\centering
\begin{tabular}{llll} \hline
Format         & Settings                                               & RMSE (Pa) & File size (bytes)  \\ \hline
32-bit floats  & No compression                                         & 0         & \num{94612}        \\
NetCDF         & Level 9 (maximum) compression                          & 0         & \num{58854}        \\
GRIB2          & Simple packing                                         & 0         & \num{41572}        \\
GRIB2          & JPEG~2000 compression                                  & <1        & \num{20450}        \\
BinCon         & Isobar spacing: \SI{200}{\Pa}, Tolerance: \ang{0.125}  & 25        & \num{1845}         \\\hline
\end{tabular}
\caption{\label{tab:compression_comparison}Comparison of file sizes for commonly used
data formats.}
\end{table*}

The BinCon format is between 11 and 51 times smaller than the other formats in
Table~\ref{tab:compression_comparison}. However, it must be borne in mind that
BinCon is a lossy compression format, whereas the others are lossless, or
nearly lossless in the case of the JPEG~2000 compression. Whether the size for
loss of information trade-off is worthwhile depends on how the forecasts are
used.

\subsection*{Other fields}
\label{sec:other_fields}

The examples presented so far have been mean sea level pressure fields.
However, the compression method works for any meteorological field which can be
contoured. For example, Figure~\ref{fig:rain} shows the \SI{24}{\hour} rainfall
accumulation forecast valid at 1200\,UTC, 5 January 2023. This also demonstrates
that the contours need not be equally spaced. In this example, the isopleths are
at 1, 2, 5, 10, 20, 50 and \SI{100}{\kg\per\m\squared}. The tolerance is
\ang{0.125}.The compressed file size is \SI{6785}{\byte}; this is larger than
the mean sea level pressure files with the same tolerance because there are
more contours for the rainfall field than the mean sea level pressure field.

\begin{figure*}
\centering
\includegraphics[width=11.7cm]{figures/apcpsfc.pdf}
\caption{\label{fig:rain}\SI{24}{\hour} rainfall accumulation.}
\end{figure*}

Variables derived from raw NWP fields can also be compressed.
Figure~\ref{fig:swp} shows forecasts of the \SI{925}{\hecto\Pa} streamlines and
wind speed covering the Southwest Pacific. These were derived from the
\SI{925}{\hecto\Pa} west-east and south-north wind speed components. For both
the streamlines and wind speed, the tolerance is \ang{0.25}. The size of the
compressed wind speed data is \SI{9304}{\byte}, and the compressed streamlines
are \SI{2076}{\byte} in size. The wind speed data are larger because the wind
speed contours are considerably more complex than the streamlines.

\begin{figure*}
\centering
\includegraphics[width=11.7cm]{figures/swp.pdf}
\caption{\label{fig:swp}\SI{24}{\hour} forecast of \SI{925}{\hecto\Pa}
stream function and wind speed for the Southwest Pacific valid
for 1200\,UTC, 5 January 2023. Wind speed units are knots.}
\end{figure*}

\section*{Conclusion}
\label{sec:conclusion}

Very high data compression ratios can be achieved by encoding data as contours
instead of grid points. Compression ratios of up to fifty or more compared to
uncompressed data are achievable. Such extreme compression methods are
potentially of use in remote locations, particularly for marine applications.
They may also have a role to play in compressing data for long-term storage.

There may be a use for high compression data products in natural disasters and
other emergencies. For example, in January 2022 communications to the
entire nation of Tonga were severely disrupted for over a month when the sole
submarine cable to the islands was severed by the Hunga Tonga-Hunga Ha\f apai
volcanic eruption~\citep{dominey-howes2022}. \citeauthor{dominey-howes2022}
recommended diversifying the way we communicate, such as by using more
satellite-based systems and other technologies. Some alternative
technologies for use in emergencies, such as high-frequency radio, are low
bandwidth where data compression techniques are useful.

A Julia language implementation of the BinCon compression method is available
at \url{https://github.com/timo007/bincon}. Readers are free to download,
experiment with and modify the code.

\section*{Acknowledgements}
\label{sec:acknowledgements}

The author would like to thank their colleagues at the Bureau of Meteorology
for discussions on the techniques presented in this article. Bob McDavitt,
retired Weather Ambassador from the Meteorological Service of New Zealand,
provided assistance with the OpenCPN software to decode and display IAC~FLEET.
The graphics in this article were created using the Generic Mapping
Tools~\citep{wessel_etal2019}.

\raggedright
\begin{thebibliography}{00}

\bibitem[Burton and Cruz, 2020]{burton_cruz2020}
\textbf{Burton P, Cruz BH.}
2020.
Data archive growth: Escaping from the black hole.
\textsl{ECMWF Newsletter} \textbf{162:} 14.

\bibitem[Dominey-Howes, 2022]{dominey-howes2022}
\textbf{Dominey-Howes D.}
2022.
The Tonga volcanic eruption has revealed the vulnerabilities in our global
telecommunication system.
\textsl{Geography Bulletin} \textbf{54(1):} 52--53.

\bibitem[Douglas and Peucker, 1973]{douglas_peucker1973}
\textbf{Douglas DH, Peucker TK.}
1973.
Algorithms for the reduction of the number of points
required to represent a digitized line or its caricature,
\textsl{Cartographica: The International Journal for Geographic Information and
Geovisualization} \textbf{10(2):} 112--122.

\bibitem[FMS, 2023]{fms2023}
\textbf{Fiji Meteorological Service.}
2023.
IAC~FLEET bulletin for the Southwest Pacific (updated daily).
\url{https://tgftp.nws.noaa.gov/data/raw/as/asps20.nffn..txt}
[accessed 4 January 2023].

\bibitem[Gertz and Grappel, 1994]{gertz_grappel1994}
\textbf{Gertz JL, Grappel RD.}
1994.
Storage and transmission of compressed weather maps and the like.
\textsl{United States Patent}, Patent Number 5,363,107.

\bibitem[Karthaus, 2012]{karthaus2012}
\textbf{Karthaus, M.}
2012.
Javascript implementation of the Ramer Douglas Peucker Algorithm.
\url{https://karthaus.nl/rdp/}
[accessed 14 January 2023].

\bibitem[Mahapatra and Makkapati, 2005]{mahapatra_makkapati2005}
\textbf{Mahapatra PR, Makkapati VV.}
2005,
Ultra high compression for weather radar reflectivity data storage and transmission.
\textsl{21st International Conference on Interactive Information Processing Systems (IIPS)
for Meteorology, Oceanography, and Hydrology.}
\url{https://ams.confex.com/ams/pdfpapers/82973.pdf}
[accessed 14 January 2023].

\bibitem[NOAA, 2023]{noaa2023}
\textbf{National Oceanographic and Atmospheric Administration.}
2023.
IAC~FLEET bulletin for the North Atlantic and Europe (updated six-hourly).
\url{https://tgftp.nws.noaa.gov/data/raw/as/asxx21.egrr..txt} 
[accessed 5 January 2023].

\bibitem[NOAA, 2021]{noaa2021}
\textbf{National Oceanographic and Atmospheric Administration.}
2021.
wgrib2: wgrib for GRIB-2.
\url{https://www.cpc.ncep.noaa.gov/products/wesley/wgrib2/}
[accessed 13 January 2023].

\bibitem[OpenCPN, 2023]{opencpn2023}
\textbf{OpenCPN development team.}
2023.
OpenCPN Chart Plotter Navigation.
\url{https://www.opencpn.org/}
[accessed 14 January 2023].

\bibitem[Organisation Météorologique Internationale, 1947]{imo1947}
\textbf{Organisation Météorologique Internationale.}
1947.
Conférence Des Directeurs, Rapport Final.
Washington, United States of America. [in French].

\bibitem[Ramer, 1972]{ramer1972}
\textbf{Ramer U.}
1972.
An Iterative Procedure for the Polygonal Approximation of Plane Curves.
\textsl{Computer Graphics and Image Processing} \textbf{1}: 244--256.

\bibitem[Scarmana, 2011]{scarmana2011}
\textbf{Scarmana G.}
2011.
Compressing Images Using Contours.
\textsl{Proceedings of the Surveying \& Spatial Sciences Biennial Conference 2011}
pp. 269-277.

\bibitem[Smith and Wessel, 1990]{smith_wessel1990}
\textbf{Smith WHF, Wessel P.}
1990.
Gridding with continuous curvature splines in tension.
\textsl{Geophysics} \textbf{55:} 293--305.

\bibitem[Wessel \textsl{et al.}, 2019]{wessel_etal2019}
\textbf{Wessel P, Luis JF, Uieda L, Scharroo R, Wobbe F, Smith WHF, Tian D.}
2019.
The Generic Mapping Tools version 6.
\textsl{Geochemistry, Geophysics, Geosystems} \textbf{20:} 5556--5564.

\bibitem[WMO, 2019]{wmo2019a}
\textbf{World Meteorological Organization.}
2019a.
\textsl{Manual on Codes - International Codes, Volume I.1, Annex II to the WMO
Technical Regulations: Part A -- Alphanumeric Codes.}
Geneva, Switzerland.

\bibitem[WMO, 2019]{wmo2019b}
\textbf{World Meteorological Organization.}
2019b.
\textsl{Manual on Codes - International Codes, Volume I.2, Annex II
to the WMO Technical Regulations: Part B -- Binary Codes, Part C -- Common
Features to Binary and Alphanumeric Codes.}
Geneva, Switzerland.

\bibitem[WMO, 2014]{wmo2014}
\textbf{World Meteorological Organization.}
2014.
Register: GRIB2 codes and flags,
\url{http://codes.wmo.int/grib2/codeflag/}
[accessed 13 January 2023].

\end{thebibliography}
   
\end{document}
